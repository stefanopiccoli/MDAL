%PREAMBOLO
\documentclass[a4paper, 12pt]{report}
\usepackage[italian]{babel}
\usepackage{graphicx}
\usepackage{amsmath,amssymb}
\usepackage{amsbsy}
\usepackage{xcolor}
\usepackage{enumitem}
\usepackage{multicol}
\renewcommand{\footnoterule}{
  \kern -3pt
  \hrule width \textwidth height 1pt
  \kern 2pt
}%ALLUNGA LINEA PIE DI PAGINA
\setcounter{tocdepth}{3}%AGGIUNGE SUBSUBSECTION ALL'INDICE
%INIZIO
\begin{document}
\title{
\textbf{Matematica Discreta}}
\author{Stefano Piccoli}
\date{\today}
\maketitle
\tableofcontents
	\chapter*{Introduzione}
	\addcontentsline{toc}{chapter}{Introduzione}
    \chapter{Algoritmo di Euclide}
    \paragraph{}Siano $a$ e $b$ due interi con $0 \leq b \leq a$
    \begin{enumerate}
        \item Se $b=0$ allora $MCD(a,b)=a$ e l'algoritmo \textbf{termina}
        \item Se $b\neq 0$ faccio la \textbf{divisione euclidea} tra $a$ e $b$:$$\boldsymbol{a=b \cdot q + r}, \; \; 0 \leq r \leq b$$
            \subitem Termina quando si trova $b=0 \rightarrow MCD(a,b)$ è l'\textbf{ultimo resto non nullo}
        \item Sostituisco $a=b$ e $b=r$ e riparto da \textbf{1}         
    \end{enumerate}
    \paragraph{Esempio} $a=168, b=132$
    \begin{align*}
        168=132\cdot 1+\textcolor{blue}{36} \rightarrow\\
        132=36 \cdot 3 +\textcolor{green}{24} \rightarrow\\
        36=24 \cdot 1+\textcolor{red}{12} \rightarrow\\
        24=12\cdot 2+0
    \end{align*}
    

    \chapter{Identità di Bezout}
    \paragraph{}Siano $a$ e $b$ due interi e $d=MCD(a,b)$ allora esistono $\boldsymbol{x}$ e $\boldsymbol{y}$ tali che:$$a\boldsymbol{x}+b\boldsymbol{y}=d$$
\end{document}