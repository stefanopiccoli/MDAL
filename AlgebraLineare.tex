%PREAMBOLO
\documentclass[a4paper, 12pt]{report}
\usepackage[italian]{babel}
\usepackage{graphicx}
\usepackage{amsmath,amssymb}
\usepackage{amsbsy}
\usepackage{xcolor}
\usepackage{enumitem}
\usepackage{multicol}
\renewcommand{\footnoterule}{
  \kern -3pt
  \hrule width \textwidth height 1pt
  \kern 2pt
}%ALLUNGA LINEA PIE DI PAGINA
%INIZIO
\begin{document}
\title{
\textbf{Algebra Lineare}}
\author{Stefano Piccoli}
\date{\today}
\maketitle
\tableofcontents
	\chapter*{Introduzione}
	\addcontentsline{toc}{chapter}{Introduzione}
        \paragraph{}L'\textbf{Algebra Lineare} si occupa di trovare soluzioni ad equazioni e sistemi \textbf{lineari}.
            \begin{center} 
                $$\begin{cases}
                    E1: x+y=3\\
                    E2: x+2y=5
                \end{cases}$$            
                $E2-E1: \textbf{y=5-3=2}$\\
                Sostituzione: $\textbf{x=1}$
            
                $$\begin{cases}
                    E1: x+y=3\\
                    E2: 2x+2y=6
                \end{cases}$$
                $E2-E1:0=0$\\
                \textbf{Hanno le stesse soluzioni (infinità)}\\
            
                $$\begin{cases}
                    E1: x+y=3\\
                    E2: 2x+2y=5
                \end{cases}$$
                $E2-E1: 0=-1$\\
                \textbf{Nessuna soluzione comune} 
            \end{center}
        \paragraph{}Quindi abbiamo 1, $\infty$ o 0 soluzioni comuni. Così sarà in generale. 
        \section{Equazioni a 3 variabili}
            \paragraph{}Le soluzioni comuni di 3 equazioni lineari a 3 variabili corrispondono all'intersezione di 3 piani nello spazio tridimensionale.
            L'intersezione può essere di 3 tipi:
            \begin{itemize}
                \item Un punto (\textbf{unica soluzione})
                \item Una retta o un piano
                \item 0 (\textbf{$\infty$ soluzioni})
            \end{itemize}
        \section{Caso generale}
            \paragraph{}Un sistema di n equazioni lineari a m variabili.
            $$\begin{cases}
                a_{11}x_1+a_{12}x_2+\dots+a_{1m}x_m=b_1\\
                a_{12}x_1+a_{22}x_2+\dots+a_{2m}x_m=b_2\\
                \vdots\\
                a_{n1}x_1+a_{n2}x_2+\dots+a_{nm}x_m=b_m
            \end{cases}$$
            \begin{center}
                $a_{ij},b_i\in\Re$\\
                $n,m > 0$\\
            \end{center}
                \subsection{Sistema omogeneo}   
                    \paragraph{}Il sistema (E) è \textbf{omogeneo} se $b_1=b_2=$\dots$=b_n=0$ 
                    $$\begin{cases}
                        a_{11}x_1+a_{12}x_2+\dots+a_{1m}x_m=0\\
                        a_{12}x_1+a_{22}x_2+\dots+a_{2m}x_m=0\\
                        \vdots\\
                        a_{n1}x_1+a_{n2}x_2+\dots+a_{nm}x_m=0
                    \end{cases}$$
                \subsection{Sistema omogeneo associato}
                    \paragraph{}Un sistema omogeneo associato è un sistema dove la parte prima parte è uguale
                    ad un altro e i coefficienti dopo l'uguale sono \textbf{0}.
                \subsection{Soluzione di un sistema}
                    \paragraph{}\textbf{Soluzione di un sistema = soluzione di un caso particolare + soluzione dell'omogenea associata}.
                    \paragraph{Esempio} $2x+3y=5$, $n=1, m=2$
                        \subparagraph{Soluzione particolare}
                            \begin{align*}
                                &2x+3y=5\\
                                &x=y=1
                            \end{align*}
                        \subparagraph{Soluzione omogenea}
                            \begin{align*}
                                &2x+3y=0\\
                                &x=-\frac{3}{2}y
                            \end{align*}
                        \subparagraph{Soluzione generale}Definiamo s parametro nel ruolo di y.
                            \begin{align*}
                                &x=1+(-\frac{3}{2})s\\
                                &y=1+s
                            \end{align*}
                \subsection{Trovare soluzioni comuni}
                    \paragraph{}Per trovare soluzioni comuni di E è necessario semplificare.
                    Le 3 operazioni utili per semplificare sono:
                            \begin{enumerate}[label=\Alph*)]
                                \item Moltiplicare un'equazione $E_i$ per una costante. $\lambda \neq 0$. $E_i\Rightarrow\lambda E_i$ 
                                \item Moltiplicare un'equazione $E_i$ per $\lambda \neq 0$ e fare la somma con $E_j$.\\ $Ej\Rightarrow E_j+\lambda E_i$. 
                                \item Scambiare due equazioni.
                            \end{enumerate}
    \chapter{Matrici}
            \paragraph{}Per semplificare inseriamo i coefficienti delle equazioni in una \textbf{matrice $n\cdot m$}.
            $$
            \begin{bmatrix}
                a_{11} & a_{12} & \dots & a_{1m}\\
                a_{21} & a_{22} & \dots & a_{2m}\\
                \vdots\\
                a_{n1} & a_{n2} & \dots & a_{nm}
            \end{bmatrix}
            $$
            \subsection{Operazioni}
                Le operazioni che potevamo usare per semplificare il sistema possiamo utilizzarle anche sulle matrici:
                    \begin{enumerate}[label=\Alph*)]
                        \item Moltiplicare una riga per $\lambda \neq 0$. $R_i \Rightarrow \lambda \cdot R_i$.
                        \item Sostituire la riga $R_j$ con una somma. $R_j \Rightarrow R_j + \lambda \cdot R_i$.
                        \item Scambiare due righe.
                    \end{enumerate}
            \section{Matrice a scalini}
                \paragraph{}Una matrice $n \cdot m$  è detta a \textbf{a scalini} se:
                    \begin{enumerate}
                        \item Le righe sono \textbf{in fondo}.
                        \item Il primo elemento di ogni riga, se esiste, è \textbf{a destra} del primo elemento $\neq 0$ della riga precedente. Un tale elemento è detti \textbf{Pivot}.
                    \end{enumerate}
                        \begin{center}
                        $
                        \begin{bmatrix}
                            1 & 1 & 1\\
                            1 & 0 & 0\\
                            0 & 0 & 1
                        \end{bmatrix}
                        NO
                        $
                        $
                        \begin{bmatrix}
                            1 & 1 & 1\\
                            0 & 1 & 1\\
                            0 & 0 & 1
                        \end{bmatrix}
                        $
                        SI
                        $
                        \begin{bmatrix}
                            0 & 1 & 1\\
                            1 & 1 & 0\\
                            0 & 0 & 1
                        \end{bmatrix}
                        $
                        NO
                    \end{center}
            \section{Algoritmo di Gauss}
                \begin{enumerate}
                    \item Se la matrice è gia in forma a \textbf{scalini} si termina. \textbf{END}.
                    \item Si cerca il primo elemento $\neq 0$ della prima colonna $\neq 0$.
                    \item Scambiando le righe possiamo supporre che questo elemento è il \textbf{pivot} della prima riga. Lo segniamo con \textit{p}. 
                    \item Se siamo in forma a scalini si \textbf{termina}. \textbf{END}.
                    \item Si annullano tutti gli elementi della colonna di \textit{p} con operazioni di tipo $R_j \Rightarrow R_j + \lambda \cdot R_i$.
                    \item Se siamo in forma a scalini si \textbf{termina}. \textbf{END}.
                    \item Si ricomincia con la matrice ottenuta \textbf{escludendo} la prima riga.
                \end{enumerate}
                \subparagraph{Esempio}
                $$
                \begin{bmatrix}
                    \textcolor{orange}{1} & -1 & 0 & 3\\
                    3 & -1 & 1 & 10\\
                    1 & 5 & 2 & 1
                \end{bmatrix}
                $$
                \paragraph{}Il \textbf{pivot} della prima riga è 1, ora devo annullare tutti gli elementi della colonna del pivot.
                $$
                    \xrightarrow[]{R_2-3R_1}                    
                    \begin{bmatrix}
                        \textcolor{orange}{1} & -1 & 0 & 3\\
                        \textcolor{blue}{0} & 2 & 1 & 1\\
                        1 & 5 & 2 & 1
                    \end{bmatrix}
                    \xrightarrow[]{R_3-R_1}                    
                    \begin{bmatrix}
                        \textcolor{orange}{1} & -1 & 0 & 3\\
                        \textcolor{blue}{0} & 2 & 1 & 1\\
                        \textcolor{blue}{0} & 6 & 2 & -2
                    \end{bmatrix}
                $$
                La prima riga è \textbf{completata}, si ripete l'algoritmo escludendola.
                $$
                    \begin{bmatrix}
                        \textcolor{orange}{1} & -1 & 0 & 3\\
                        \hline
                        \textcolor{blue}{0} & \textcolor{orange}{2} & 1 & 1\\
                        \textcolor{blue}{0} & 6 & 2 & -2
                    \end{bmatrix}
                $$
                Nella seconda riga il \textbf{pivot} è 2, si procede annullando le colonne sotto il pivot.
                $$
                    \xrightarrow[]{R_3-R_1}
                    \begin{bmatrix}
                        \textcolor{orange}{1} & -1 & 0 & 3\\
                        \hline
                        \textcolor{blue}{0} & \textcolor{orange}{2} & 1 & 1\\
                        \textcolor{blue}{0} & 6 & 2 & -2
                    \end{bmatrix}
                    \xrightarrow[]{R_3-3R_2}
                    \begin{bmatrix}
                        \textcolor{orange}{1} & -1 & 0 & 3\\
                        \hline
                        \textcolor{blue}{0} & \textcolor{orange}{2} & 1 & 1\\
                        \textcolor{blue}{0} & \textcolor{blue}{0} & -1 & 5
                    \end{bmatrix}
                $$
                La seconda riga è \textbf{completata}, si ripete l'algoritmo escludendola.
                $$
                \begin{bmatrix}
                    \textcolor{orange}{1} & -1 & 0 & 3\\
                    \textcolor{blue}{0} & \textcolor{orange}{2} & 1 & 1\\
                    \hline
                    \textcolor{blue}{0} & \textcolor{blue}{0} & \textcolor{orange}{-1} & 5
                \end{bmatrix}
                $$
                L'algoritmo termina poiché -1 è un \textbf{pivot} e non ci sono colonne da annullare.
                \subparagraph{Conclusioni}La matrice ritrasformata in sistema di equazioni è la seguente:
                $$
                \begin{cases}
                    x_1-x_2+3x_4=0\\
                    2x_2+x_3+x_4=0\\
                    -x_3-5x_4=0
                \end{cases}
                $$
                La \textbf{colonna di $x_4$} è senza \textbf{pivot} quindi \textbf{$x_4$} è detta \textbf{variabile libera}, e può assumere qualsiasi
                valore nel sistema.
                Sostituiamo la \textbf{variabile libera} $x_4$ con il parametro \textit{t}.
                $$
                \begin{cases}
                    x_1-x_2+3t=0\\
                    2x_2+x_3+t=0\\
                    -x_3-5t=0
                \end{cases}
                \begin{cases}
                    x_1-x_2+3t=0\\
                    2x_2+x_3+t=0\\
                    x_3=-5t
                \end{cases}
                \begin{cases}
                    x_1-x_2+3t=0\\
                    2x_2-5t+t=0\\
                    x_3=-5t
                \end{cases}
                $$
                $$
                \begin{cases}
                    x_1-x_2+3t=0\\
                    x_2=2t\\
                    x_3=-5t
                \end{cases}
                \begin{cases}
                    x_1-2t+3t=0\\
                    x_2=2t\\
                    x_3=-5t
                \end{cases}
                \begin{cases}
                    x_1=-t\\
                    x_2=2t\\
                    x_3=-5t
                \end{cases}
                $$ 
            L'equazione ha \textbf{infinite soluzioni} che possono essere parametrizzate in \textit{t}.
            \subsection{Casi possibili}
            \paragraph{}Se nella forma a scalini:
                \begin{enumerate}
                    \item \textbf{Ogni colonna} "non aggiunta" \textbf{ha un pivot} $\Leftrightarrow \exists$ \textbf{unica soluzione}
                    \item C'è un \textbf{pivot nell'ultima colonna} $\Leftrightarrow \nexists$ \textbf{soluzione}
                    \item C'è una \textbf{colonna "non aggiunta" senza pivot} e l'ultima colonna non ne ha $\Leftrightarrow \exists \; \infty$ \textbf{soluzioni} 
                \end{enumerate}
        \clearpage
        \section{Matrice ridotta a scalini}
            \paragraph{}Una matrice è in forma \textbf{ridotta} a scalini se:
                \begin{itemize}
                    \item È in forma \textbf{a scalini}
                    \item Ogni \textbf{pivot} è = 1
                    \item Ogni \textbf{pivot} è l'unico elemento $\neq 0$ nella sua colonna
                \end{itemize}
                \subparagraph{Esempi}
                $$
                \begin{bmatrix}
                    \textcolor{blue}{1} & 2 & 0 & 0\\
                    0 & 0 & \textcolor{blue}1 & 0\\
                    0 & 0 & 0 & \textcolor{blue}{1}
                \end{bmatrix}
                \text{SI }
                \begin{bmatrix}
                    \textcolor{blue}{1} & 2 & \textcolor{orange}{3} & \textcolor{orange}{4}\\
                    0 & 0 & \textcolor{blue}{1} & \textcolor{orange}{2}\\
                    0 & 0 & 0 & \textcolor{blue}{1}
                \end{bmatrix}
                \text{NO (A scalini ma non ridotta)}
                $$
        \section{Algoritmo di Gauss-Jordan}
            \paragraph{}L'algoritmo produce una matrice in forma \textbf{ridotta} a scalini attraverso operazioni
            del tipo A, B, C.
                \begin{enumerate}
                    \item Con l'\textbf{algoritmo di Gauss} si riduce a scalini la matrice.
                    \item Nelle colonne dei pivot gli elementi della colonna superiore e a sinistra nella riga sono già = 0.
                    \textbf{Annullare} gli elementi sopra il pivot nella colonna con \textbf{operazioni del tipo B} ($R_j \Rightarrow R_j + \lambda \cdot R_i$).
                    \item In ogni riga si \textbf{cerca il pivot} (se esiste). Se il pivot $\lambda \neq 1$, si moltiplica la riga per $\frac{1}{\lambda}$.
                \end{enumerate}
                \subparagraph{Esempio}Partiamo da una matrice già ridotta a scalini dall'algoritmo di Gauss.
                    $$
                    \begin{bmatrix}
                        2 & 1 & -1 & \vline & -1\\
                        3 & 2 & -1 & \vline & 0\\
                        4 & -3 & 1 & \vline & -1\\
                        5 & -2 & 2 & \vline & 2
                    \end{bmatrix}
                    \xrightarrow[]{\text{Algoritmo di Gauss}}
                    \begin{bmatrix}
                        \textcolor{blue}{2} & 1 & -1 & \vline & -1\\
                        0 & \textcolor{blue}{1} & 1 & \vline & 3\\
                        0 & 0 & \textcolor{blue}{1} & \vline & 2\\
                        0 & 0 & 0 & \vline & 0
                    \end{bmatrix}
                    $$ 
                    Ora applichiamo l'\textbf{algoritmo di Gauss-Jordan} alla matrice a scalini per trasformarla in matrice \textbf{ridotta} a scalini.
                    $$
                    \begin{bmatrix}
                        \textcolor{blue}{2} & 1 & -1 & \vline & -1\\
                        0 & \textcolor{blue}{1} & 1 & \vline & 3\\
                        0 & 0 & \textcolor{blue}{1} & \vline & 2\\
                        0 & 0 & 0 & \vline & 0
                    \end{bmatrix}
                    $$
                    Si \textbf{azzerano} gli elementi nelle colonne dei pivot che sono $\neq$ 0.
                    $$
                    \begin{bmatrix}
                        2 & \textcolor{orange}{1} & \textcolor{orange}{-1} & \vline & -1\\
                        0 & 1 & \textcolor{orange}{1} & \vline & 3\\
                        0 & 0 & 1 & \vline & 2\\
                        0 & 0 & 0 & \vline & 0
                    \end{bmatrix}
                    \xrightarrow[]{R_1-R_2}
                    \begin{bmatrix}
                        2 & \textcolor{blue}{0} & \textcolor{orange}{-2} & \vline & -4\\
                        0 & 1 & \textcolor{orange}{1} & \vline & 3\\
                        0 & 0 & 1 & \vline & 2\\
                        0 & 0 & 0 & \vline & 0
                    \end{bmatrix}
                    \xrightarrow[R_2-R_3]{R_1+2R_3}
                    \begin{bmatrix}
                        2 & \textcolor{blue}{0} & \textcolor{blue}{0} & \vline & 0\\
                        0 & 1 & \textcolor{blue}{0} & \vline & 1\\
                        0 & 0 & 1 & \vline & 2\\
                        0 & 0 & 0 & \vline & 0
                    \end{bmatrix}
                    $$
                    Ora nelle colonne dei pivot \textbf{tutti gli elementi sono = 0} eccetto il pivot.
                    Si individuano i \textbf{pivot} $\neq 1$ e si procede con la loro \textbf{trasformazione a 1}.
                    Si moltiplicano le righe con i \textbf{pivot} $\neq$ 1 per il loro \textbf{reciproco}.
                    $$
                    \begin{bmatrix}
                        \textcolor{orange}{2} & 0 & 0 & \vline & 0\\
                        0 & 1 & 0 & \vline & 1\\
                        0 & 0 & 1 & \vline & 2\\
                        0 & 0 & 0 & \vline & 0
                    \end{bmatrix}
                    \xrightarrow[]{R_1\rightarrow \frac{1}{2}R_1}
                    \begin{bmatrix}
                        \textcolor{blue}{1} & 0 & 0 & \vline & 0\\
                        0 & 1 & 0 & \vline & 1\\
                        0 & 0 & 1 & \vline & 2\\
                        0 & 0 & 0 & \vline & 0
                    \end{bmatrix}
                    $$
                \subparagraph{Conclusioni}
                $$
                \begin{cases}
                x_1=0\\
                x_2=1\\
                x_3=2  
                \end{cases}
                $$
    \chapter{Spazi vettoriali}
        \paragraph{}Si parla di \textbf{spazi vettoriali} quando definiamo punti e vettori
        nel piano $\mathbb{R}^2$. Un \textbf{punto} di $\mathbb{R}^2$ si può descrivere con \textbf{due coordinate}
        $(x_1,x_2)$, ma anche con un \textbf{vettore} (una freccia) dall'\textbf{origine} $(0,0)$ a $(x_1,x_2)$ 
            \subsection{Somma}
            \paragraph{}Si può fare la \textbf{somma} di due vettori:
                \begin{itemize}
                    \item Sulle \textbf{coordinate}: $(x_1,x_2)+(x'_1+x'_2):=(x_1+x'_1,x_2+x'_2)$
                    \item \textbf{Geometricamente}: \textbf{Legge del parallelogramma} dove la \textbf{diagonale del parallelogramma} è la somma dei due vettori.
                \end{itemize}
            \subsection{Moltiplicazione}
            \paragraph{}Un vettore può essere moltiplicato con uno scalare $\lambda \in \mathbb{R}$.
            \begin{itemize}
                \item Sulle \textbf{coordinate}: $\lambda (x_1,x_2):=(\lambda x_1,\lambda x_2)$
                \item \textbf{Geometricamente}: La \textbf{lunghezza} è moltiplicata da $\lambda$ ma l'angolo non cambia.
            \end{itemize}
        \clearpage
        \section{Spazi vettoriali di dimensione n}
            \paragraph{}Si definisce $\mathbb{R}^n:= \left \{
            \begin{bmatrix}
            x_1\\
            x_2\\
            \vdots\\
            x_n    
            \end{bmatrix}
            : x_i \in \mathbb{R}
            \right \}
            $
            uno \textbf{spazio n-dim standard} o spazio dei vettori colonna.\\
            Un spazio vettoriale di dimensione \textbf{2} corrisponde ad un \textbf{piano}, di dimensione \textbf{3} ad uno
            spazio \textbf{euclideiano}.
            \paragraph{Definizione} Uno \textbf{spazio vettoriale} su $\mathbb{R}$ è un insieme V che ammette due tipi di operazioni:
                \begin{itemize}
                    \item \textbf{Somma}: $v_1,v_2 \in V \rightarrow v_1+v_2 \in V$.
                    \item \textbf{Prodotto} con $\lambda \in \mathbb{R}$ : $v \in V \rightarrow \lambda \cdot v \in V$. 
                \end{itemize}
            Le operazioni devono soddisfare:
                \begin{multicols}{2}
                \begin{enumerate}              
                    \item $(v_1+v_2)+v_3 = v_1+(v_2+v_3)$
                    \item $v_1+v_2 = v_2+v_1$
                    \item \footnote{$\exists !$= Esiste un unico}$\exists ! 0 \in V: 0+v=v+0=v \; \forall v$
                    \item $\forall v \; \exists ! -v \in V: v+(-v)=(-v)+v=0$
                    \columnbreak
                    \item $(\lambda_1+\lambda_2)\cdot v = \lambda_1 \cdot v +\lambda_2 \cdot v$
                    \item $\lambda \cdot(v_1+v_2)=\lambda \cdot v_1+\lambda \cdot v_2$
                    \item $(\lambda_1 \cdot \lambda_2)\cdot v = \lambda_1 \cdot(\lambda_2 \cdot v)$
                    \item $1 \cdot v = v$
                \end{enumerate}
                \end{multicols}
            \subsection{Somma}
                $$
                \begin{bmatrix}
                    x_1\\
                    x_2\\
                    \vdots\\
                    x_n    
                \end{bmatrix}
                +
                \begin{bmatrix}
                    x'_1\\
                    x'_2\\
                    \vdots\\
                    x'_n    
                \end{bmatrix}
                :=
                \begin{bmatrix}
                    x_1+x'_1\\
                    x_2+x'_2\\
                    \vdots\\
                    x_n+x'_n    
                \end{bmatrix}
                $$
            \subsection{Moltiplicazione}
                $$
                \lambda \cdot
                \begin{bmatrix}
                    x_1\\
                    x_2\\
                    \vdots\\
                    x_n    
                \end{bmatrix}
                := 
                \begin{bmatrix}
                    \lambda \cdot x_1\\
                    \lambda \cdot x_2\\
                    \vdots\\
                    \lambda \cdot x_n    
                \end{bmatrix}
                \lambda \in \mathbb{R}.
                $$
            \section{Sottospazi vettoriali}
                \paragraph{}Sia V uno spazio vettoriale. Un \textbf{sottospazio} $W \subset V$ è
                un sottoinsieme tale che
                    \begin{itemize}
                        \item Dati due vettori nel sottospazio, la loro somma sarà nel sottospazio. $$v_1,v_2 \in W \Rightarrow v_1+v_2 \in W$$
                        \item Dato un vettore nel sottospazio, il prodotto con un qualsiasi scalare è contenuto nel sottospazio. $$v \in W \Rightarrow \lambda v \in W \; \forall \lambda$$
                    \end{itemize} 
                Un sottospazio $\boldsymbol{W \subset V}$ \textbf{è uno spazio vettoriale}.
                \subparagraph{Esempio}
                    \begin{enumerate}
                        \item $ \left \{
                                \begin{bmatrix}
                                t_1\\
                                t_2
                                \end{bmatrix}
                                \in \mathbb{R}^2 : t_1+t_2=0
                                \right \}\subset \mathbb{R}^2
                                $ 
                                è un sottospazio.\\[2pt]
                                In generale
                                $$ \left \{
                                \begin{bmatrix}
                                t_1\\
                                t_2\\
                                \vdots\\
                                t_m
                                \end{bmatrix}
                                \in \mathbb{R}^m : 
                                \begin{cases}
                                    a_{11}t_1+a_{12}t_2+\dots+a_{1m}t_m=0\\
                                    a_{21}t_1+a_{22}t_2+\dots+a_{2m}t_m=0\\
                                    \vdots\\
                                    a_{n1}t_1+a_{n2}t_2+\dots+a_{nm}t_m=0
                                \end{cases}
                                \right \}\subset \mathbb{R}^m
                                $$
                                è \textbf{sottospazio}.\\
                                Quindi le \textbf{soluzioni} di un \textbf{sistema di equazioni lineari omogenei} a n
                                variabili definiscono un \textbf{sottospazio} di $\mathbb{R}^m$.\\
                                \textbf{Non definiscono un sottospazio} di $\mathbb{R}^m$ le soluzioni di equazioni lineari \textbf{non omogenee} (coefficiente $\neq 0$).
                    \end{enumerate}
    \chapter{Combinazioni lineari}
                \paragraph{}Sia V uno spazio vettoriale, $v_1,v_2,\dots,v_m \in V$.
                Una \textbf{combinazione lineare} di $v_1,\dots,v_m$ è una somma $\boldsymbol{\lambda_1 v_1+\lambda_2 v_2+\dots+\lambda_m v_m \in V}$, dove $\lambda_1,\lambda_2,\dots,\lambda_m \in \mathbb{R}$.\\
                La combinazione lineare è detta \textbf{banale} se $\lambda_1=\dots=\lambda_m=0$.
                \subparagraph{Esempio}
                    $$
                    V=\mathbb{R}^2,\;v_1=
                    \begin{bmatrix}
                        1\\
                        1
                    \end{bmatrix}
                    ,\;v_2=
                    \begin{bmatrix}
                        2\\
                        2
                    \end{bmatrix}
                    $$
                    \subparagraph{}Allora $-2v_1+1v_2=0$ è \textbf{combinazione lineare non banale}.
        \section{Span}
            \paragraph{}Siano $v_1,\dots,v_m \in V \;m$ vettori. Il \textbf{sottospazio generato} da $v_1,\dots,v_m$ è:
            $$
                Span(v_1,v_2,...,v_m):= \left \{\lambda_1v_1+\lambda_2v_2+\dots+\lambda_mv_m\;:\;\lambda_1,\dots,\lambda_m \in \mathbb{R}\right\}
            $$   
            Quindi \textit{Span} è l'insieme di \textbf{tutte le combinazioni lineari}.\\
            $Span(v_1,v_2,\dots,v_m) \subset V$ è un \textbf{sottospazio}.     
                \subparagraph{Esempi}
                \begin{enumerate}
                    \item 
                        $$
                        \mathbb{R}^2 = Span \left\{
                            \begin{bmatrix}
                                0\\
                                1
                            \end{bmatrix}
                            ,
                            \begin{bmatrix}
                                1\\
                                0
                            \end{bmatrix}
                            \right\}
                        $$ 
                        $
                        Span \left\{
                        \begin{bmatrix}
                            0\\
                            1
                        \end{bmatrix}
                        \right\},Span
                        \left\{
                        \begin{bmatrix}
                            1\\
                            0
                        \end{bmatrix}
                        \right\} \subset \mathbb{R}^2
                        $    
                        sono due rette, rispettivamente dell'asse x e y.
                    \item 
                        $$
                        W := \left\{
                            \begin{bmatrix}
                                t_1\\
                                t_2\\
                                t_3
                            \end{bmatrix}
                            \in \mathbb{R}^3 : t_1=0 \right\}
                        $$
                        Allora
                        $
                        W=Span \left\{
                            \begin{bmatrix}
                                0\\
                                1\\
                                0
                            \end{bmatrix}
                            ,
                            \begin{bmatrix}
                                0\\
                                0\\
                                1
                            \end{bmatrix}
                            \right\}=Span \left\{
                                \begin{bmatrix}
                                    0\\
                                    1\\
                                    1
                                \end{bmatrix}
                                ,
                                \begin{bmatrix}
                                    0\\
                                    0\\
                                    -1
                                \end{bmatrix}
                                \right\}
                        $.\\
                        Quindi un \textbf{sottospazio} può essere lo \textbf{span di vettori diversi}.
                \end{enumerate}
                \paragraph{Verificare che $\boldsymbol{Span(v_1,v_2,v_3)=Span(v_1,v_2,v_3,v_4)=\mathbb{R}^3}$}
                    $$
                    v_1=
                    \begin{bmatrix}
                        1\\
                        2\\
                        3\\
                    \end{bmatrix},
                    v_2=
                    \begin{bmatrix}
                        1\\
                        0\\
                        1
                    \end{bmatrix},
                    v_3=
                    \begin{bmatrix}
                        0\\
                        0\\
                        1
                    \end{bmatrix},
                    v_4=
                    \begin{bmatrix}
                        2\\
                        2\\
                        4
                    \end{bmatrix}
                    $$
                    Se 
                    $
                    v=
                    \begin{bmatrix}
                        b_1\\
                        b_2\\
                        b_3\\
                    \end{bmatrix}
                    \in \mathbb{R}^3
                    $
                    applicando l'\textbf{Algoritmo di Gauss} si ottiene:
                    $$
                    \begin{bmatrix}
                        1 & 1 & 0 & \vline & b_1\\
                        2 & 0 & 0 & \vline & b_2\\
                        3 & 1 & 1 & \vline & b_3
                    \end{bmatrix}
                    \xrightarrow[R_3-3R_1]{R_2-2R_1}
                    \begin{bmatrix}
                        \textcolor{blue}{1} & 1 & 0 & \vline & b_1\\
                        0 & -2 & 0 & \vline & b_2-2b_1\\
                        0 & -2 & 1 & \vline & b_3-3b_1
                    \end{bmatrix}
                    $$
                    $$
                    \xrightarrow{R3-R2}
                    \begin{bmatrix}
                        \textcolor{blue}{1} & 1 & 0 & \vline & b_1\\
                        0 & \textcolor{blue}{-2} & 0 & \vline & b_2-2b_1\\
                        0 & 0 & \textcolor{blue}{1} & \vline & b_3-b_1-b_2
                    \end{bmatrix}
                    $$
                    3 \textbf{pivots} nelle 3 colonne a sinistra (non ci interessa a destra) quindi
                    $$
                    \begin{cases}
                        x_1+x_2=b_1\\
                        2x_1=b_2\\
                        3x_1+x_2+x_3=b_3
                    \end{cases}
                    $$
                    ammette un' \textbf{unica soluzione} $\lambda_1,\lambda_2,\lambda_3$:
                    $$
                    \lambda_1
                    \begin{bmatrix}
                        1\\
                        2\\
                        3
                    \end{bmatrix}
                    +\lambda_2
                    \begin{bmatrix}
                        1\\
                        0\\
                        1
                    \end{bmatrix}
                    +\lambda_3
                    \begin{bmatrix}
                        0\\
                        0\\
                        1
                    \end{bmatrix}
                    =
                    \begin{bmatrix}
                        b_1\\
                        b_2\\
                        b_3
                    \end{bmatrix}
                    \text{.}
                    $$
                    Il \textbf{vettore generale} v è contenuto in $Span(v_1,v_2,v_3)$.
                    \paragraph{In generale} Se $v_1,v_2,\dots,v_n \in V$ sono vettori tali che $v_n$ è \textbf{combinazione lineare}
                    di $v_1,v_2,\dots,v_{n-1} \Rightarrow Span(v_1,v_1,\dots,v_n)=Span(v_1,v_1,\dots,v_{n-1})$.
                    
        \section{Dipendenza lineare}
            \paragraph{}I vettori $v_1,v_2,\dots,v_m \in V$ sono \textbf{linearmente indipendenti} se
            $$\lambda v_1+\lambda_2 v_2+\dots+\lambda_m V_m = 0$$ vale \textbf{solo} per $\lambda_1=\dots=\lambda_m=0$.
            Altrimenti sono \textbf{linearmente dipendenti}.
            \paragraph{Geometricamente} Vettori linearmente dipendenti hanno la \textbf{stessa retta}.
            \paragraph{Proposizione} $v_1,v_2,\dots,v_m$ sono \textbf{linearmente dipendenti} $\Leftrightarrow \exists i : v_i$ è combinazione lineare dei $v_j\; \forall j\neq i$.
            \paragraph{Verificare se m vettori sono linearmente indipendenti}
                $$
                v_1=
                \begin{bmatrix}
                    a_{11}\\
                    a_{21}\\
                    \vdots\\
                    a_{n1}       
                \end{bmatrix}
                ,v_2=
                \begin{bmatrix}
                    a_{12}\\
                    a_{22}\\
                    \vdots\\
                    a_{n2} 
                \end{bmatrix}
                ,\dots,\;
                v_m=
                \begin{bmatrix}
                    a_{1m}\\
                    a_{2m}\\
                    \vdots\\
                    a_{nm} 
                \end{bmatrix}
                $$
                L'equazione $\lambda_1v_1+\lambda_2v_2+\dots+\lambda_mv_m=0$ \textbf{vale se e solo se} $(\lambda_1,\dots,\lambda_m)$ \textbf{è soluzione del sistema}
                $$
                    \begin{cases}
                        a_{11}x_1+a_{12}x_2+\dots+a_{1m}x_m = 0\\
                        a_{21}x_1+a_{22}x_2+\dots+a_{1m}x_m = 0\\
                        \vdots\\
                        a_{n1}x_1+a_{n2}x_2+\dots+a_{nm}x_m = 0
                    \end{cases}
                $$               
                dove \textbf{x sostituisce} $\boldsymbol{\lambda}$ e lo 0 dell'equazione corrisponde al vettore
                $
                \begin{bmatrix}
                    0\\
                    \vdots\\
                    0
                \end{bmatrix}
                $.\\
                Quindi $v_1,\dots,v_m$ sono \textbf{linearmente indipendenti} $\Leftrightarrow$ il sistema ammette \textbf{solo la soluzione banale}, cioè $x=(0,\dots,0)$.
                \paragraph{Esempio} Verificare che i seguenti vettori di $\mathbb{R}^3$ siano \textbf{linearmente indipendenti}.
                    $$
                    v_1=
                    \begin{bmatrix}
                        1\\
                        2\\
                        3\\
                    \end{bmatrix},
                    v_2=
                    \begin{bmatrix}
                        1\\
                        0\\
                        1
                    \end{bmatrix},
                    v_3=
                    \begin{bmatrix}
                        0\\
                        0\\
                        1
                    \end{bmatrix},
                    v_4=
                    \begin{bmatrix}
                        2\\
                        2\\
                        4
                    \end{bmatrix}
                    $$
                Dobbiamo cercare le soluzioni del sistema \textbf{lineare omogeneo} con la \textbf{matrice dei coefficienti associata}.
                    $$
                    \begin{bmatrix}
                        1 & 1 & 0 & 2\\
                        2 & 0 & 0 & 2\\
                        3 & 1 & 1 & 4
                    \end{bmatrix}
                    $$
                \textbf{Algoritmo di Gauss:}
                    $$
                    \xrightarrow[R_3-3R_1]{R_2-2R_1}
                    \begin{bmatrix}
                        \textcolor{blue}{1} & 1 & 0 & 2\\
                        0 & -2 & 0 & -2\\
                        0 & -2 & 1 & -2
                    \end{bmatrix}
                    \xrightarrow{R_3-R_2}
                    \begin{bmatrix}
                        \textcolor{blue}{1} & 1 & 0 & 2\\
                        0 & \textcolor{blue}{-2} & 0 & -2\\
                        0 & 0 & \textcolor{blue}{1} & 0
                    \end{bmatrix}
                    $$
                Ci sono 3 \textbf{pivots} e una \textbf{variabile libera} $\Rightarrow \boldsymbol{\infty}$ \textbf{soluzioni}.\\
                Il sistema ammette \textbf{soluzioni non banali} $\Rightarrow$ i vettori sono \textbf{linearmente dipendenti}.
        \section{Base}
            \paragraph{}Un sistema $v_1,\dots,v_n$ di vettori è una \textbf{base} di V se:
            \begin{itemize}
                \item i vettori $v_1,\dots,v_n$ sono \textbf{linearmente indipendenti}
                \item $\boldsymbol{Span(v_1,\dots,v_n) = V}$
            \end{itemize}
        %-33
            \paragraph{Esempio} Base standard di $\mathbb{R}^n$:
                $$
                e_1=
                \begin{bmatrix}
                    1\\
                    0\\
                    \vdots\\
                    0
                \end{bmatrix}
                ,e_2=
                \begin{bmatrix}
                    0\\
                    1\\
                    \vdots\\
                    0
                \end{bmatrix}
                ,\dots,e_n=
                \begin{bmatrix}
                    0\\
                    0\\
                    \vdots\\
                    1
                \end{bmatrix}
                $$
                Si osserva $
                    \begin{bmatrix}
                        \lambda_1\\
                        \lambda_2\\
                        \vdots\\
                        \lambda_n\\
                    \end{bmatrix}
                = \lambda_1 e_1+\lambda_2 e_2+\dots+\lambda_n e_n.
                $\\
                Dunque $Span (e_1,\dots,e_n)=\mathbb{R}^n$ e 
                $\lambda_1 e_1+\dots+\lambda_n e_n=0$ 
                se e solo se \\$\lambda_1=\dots=\lambda_n=0$.
            \subsection{Coordinate}
                \paragraph{}Sia $v_1,\dots,v_n$ una base di V e $v \in V$ un vettore. 
                Allora $$\exists! \; \alpha_1,\dots,\alpha_n : v=\alpha_1 v_1+\alpha_2 v_2+\dots+\alpha_n v_n$$
                ovvero \textbf{ogni vettore} si scrive in un modo \textbf{unico} come \textbf{combinazione lineare} degli \textbf{elementi della base}.\\
                Gli $\boldsymbol{\alpha_i}$ sono le \textbf{coordinate} di v rispetto alla \textbf{base}.
                \paragraph{Trovare le coordinate di un vettore rispetto alla base}\mbox{}\\
                Sappiamo da esercizi precedenti che
                    $
                    v_1=
                    \begin{bmatrix}
                        1\\
                        2\\
                        3
                    \end{bmatrix}
                    ,v_2=
                    \begin{bmatrix}
                        1\\
                        0\\
                        1
                    \end{bmatrix}
                    ,v_3=
                    \begin{bmatrix}
                        0\\
                        0\\
                        1
                    \end{bmatrix}
                    $
                    sono una \textbf{base} di $\mathbb{R}^3$. Trovare le coordinate di
                    $
                    \begin{bmatrix}
                        0\\
                        2\\
                        1
                    \end{bmatrix}
                    $ rispetto a questa base.
                    $$
                    \alpha_1
                    \begin{bmatrix}
                        1\\
                        2\\
                        3
                    \end{bmatrix}
                    +\alpha_2
                    \begin{bmatrix}
                        1\\
                        0\\
                        1
                    \end{bmatrix}
                    +\alpha_3
                    \begin{bmatrix}
                        0\\
                        0\\
                        1
                    \end{bmatrix}
                    =
                    \begin{bmatrix}
                        0\\
                        2\\
                        1
                    \end{bmatrix}
                    $$
                Applichiamo l'\textbf{algoritmo di Gauss-Jordan}.
                $$
                \begin{bmatrix}
                    1 & 1 & 0 & \vline & 0\\
                    2 & 0 & 0 & \vline & 2\\
                    3 & 1 & 1 & \vline & 1
                \end{bmatrix}
                \xrightarrow[R_3-3R_1]{R_2-2R_1}
                \begin{bmatrix}
                    1 & 1 & 0 & \vline & 0\\
                    0 & -2 & 0 & \vline & 2\\
                    0 & -2 & 1 & \vline & 1
                \end{bmatrix}
                \xrightarrow[R_2\rightarrow \frac{1}{2}R_2]{R_3-R_2}
                \begin{bmatrix}
                    1 & 1 & 0 & \vline & 1\\
                    0 & 1 & 0 & \vline & -1\\
                    0 & 0 & 1 & \vline & -1
                \end{bmatrix}
                $$
                $$
                \xrightarrow[]{R_1-R_2}
                \begin{bmatrix}
                    1 & 0 & 0 & \vline & 1\\
                    0 & 1 & 0 & \vline & -1\\
                    0 & 0 & 1 & \vline & -1
                \end{bmatrix}
                $$
                Quindi 
                $
                \begin{cases}
                    \alpha_1=1\\
                    \alpha_2=-1\\
                    \alpha_3=-1
                \end{cases}
                \text{e }
                1
                \begin{bmatrix}
                    1\\
                    2\\
                    3
                \end{bmatrix}
                +-1
                \begin{bmatrix}
                    1\\
                    0\\
                    1
                \end{bmatrix}
                +-1
                \begin{bmatrix}
                    0\\
                    0\\
                    1
                \end{bmatrix}
                =
                \begin{bmatrix}
                    0\\
                    2\\
                    1
                \end{bmatrix}
                $
\end{document}