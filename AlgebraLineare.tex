\documentclass[a4paper, 12pt]{report}
\usepackage[italian]{babel}
\usepackage{graphicx}
\usepackage{amsmath,amssymb}
\usepackage{xcolor}
\usepackage{enumitem}

\begin{document}
\title{
\textbf{Algebra Lineare}}
\author{Stefano Piccoli}
\date{\today}
\maketitle
\tableofcontents
	\chapter*{Introduzione}
	\addcontentsline{toc}{chapter}{Introduzione}
        \paragraph{}L'\textbf{Algebra Lineare} si occupa di trovare soluzioni ad equazioni e sistemi \textbf{lineari}.
            \begin{center} 
                $$\begin{cases}
                    E1: x+y=3\\
                    E2: x+2y=5
                \end{cases}$$            
                $E2-E1: \textbf{y=5-3=2}$\\
                Sostituzione: $\textbf{x=1}$
            
                $$\begin{cases}
                    E1: x+y=3\\
                    E2: 2x+2y=6
                \end{cases}$$
                $E2-E1:0=0$\\
                \textbf{Hanno le stesse soluzioni (infinità)}\\
            
                $$\begin{cases}
                    E1: x+y=3\\
                    E2: 2x+2y=5
                \end{cases}$$
                $E2-E1: 0=-1$\\
                \textbf{Nessuna soluzione comune} 
            \end{center}
        \paragraph{}Quindi abbiamo 1, $\infty$ o 0 soluzioni comuni. Così sarà in generale. 
        \section{Equazioni a 3 variabili}
            \paragraph{}Le soluzioni comuni di 3 equazioni lineari a 3 variabili corrispondono all'intersezione di 3 piani nello spazio tridimensionale.
            L'intersezione può essere di 3 tipi:
            \begin{itemize}
                \item Un punto (\textbf{unica soluzione})
                \item Una retta o un piano
                \item 0 (\textbf{$\infty$ soluzioni})
            \end{itemize}
        \section{Caso generale}
            \paragraph{}Un sistema di n equazioni lineari a m variabili.
            $$\begin{cases}
                a_{11}x_1+a_{12}x_2+\dots+a_{1m}x_m=b_1\\
                a_{12}x_1+a_{22}x_2+\dots+a_{2m}x_m=b_2\\
                \vdots\\
                a_{n1}x_1+a_{n2}x_2+\dots+a_{nm}x_m=b_m
            \end{cases}$$
            \begin{center}
                $a_{ij},b_i\in\Re$\\
                $n,m > 0$\\
            \end{center}
                \subsection{Sistema omogeneo}   
                    \paragraph{}Il sistema (E) è \textbf{omogeneo} se $b_1=b_2=$\dots$=b_n=0$ 
                    $$\begin{cases}
                        a_{11}x_1+a_{12}x_2+\dots+a_{1m}x_m=0\\
                        a_{12}x_1+a_{22}x_2+\dots+a_{2m}x_m=0\\
                        \vdots\\
                        a_{n1}x_1+a_{n2}x_2+\dots+a_{nm}x_m=0
                    \end{cases}$$
                \subsection{Sistema omogeneo associato}
                    \paragraph{}Un sistema omogeneo associato è un sistema dove la parte prima parte è uguale
                    ad un altro e i coefficienti dopo l'uguale sono \textbf{0}.
                \subsection{Soluzione di un sistema}
                    \paragraph{}\textbf{Soluzione di un sistema = soluzione di un caso particolare + soluzione dell'omogenea associata}.
                    \paragraph{Esempio} $2x+3y=5$, $n=1, m=2$
                        \subparagraph{Soluzione particolare}
                            \begin{align*}
                                &2x+3y=5\\
                                &x=y=1
                            \end{align*}
                        \subparagraph{Soluzione omogenea}
                            \begin{align*}
                                &2x+3y=0\\
                                &x=-\frac{3}{2}y
                            \end{align*}
                        \subparagraph{Soluzione generale}Definiamo s parametro nel ruolo di y.
                            \begin{align*}
                                &x=1+(-\frac{3}{2})s\\
                                &y=1+s
                            \end{align*}
                \subsection{Trovare soluzioni comuni}
                    \paragraph{}Per trovare soluzioni comuni di E è necessario semplificare.
                    Le 3 operazioni utili per semplificare sono:
                            \begin{enumerate}[label=\Alph*)]
                                \item Moltiplicare un'equazione $E_i$ per una costante. $\lambda \neq 0$. $E_i\Rightarrow\lambda E_i$ 
                                \item Moltiplicare un'equazione $E_i$ per $\lambda \neq 0$ e fare la somma con $E_j$.\\ $Ej\Rightarrow E_j+\lambda E_i$. 
                                \item Scambiare due equazioni.
                            \end{enumerate}
        \chapter{Matrici}
            \paragraph{}Per semplificare inseriamo i coefficienti delle equazioni in una \textbf{matrice $n\cdot m$}.
            $$
            \begin{bmatrix}
                a_{11} & a_{12} & \dots & a_{1m}\\
                a_{21} & a_{22} & \dots & a_{2m}\\
                \vdots\\
                a_{n1} & a_{n2} & \dots & a_{nm}
            \end{bmatrix}
            $$
            \subsection{Operazioni}
                Le operazioni che potevamo usare per semplificare il sistema possiamo utilizzarle anche sulle matrici:
                    \begin{enumerate}[label=\Alph*)]
                        \item Moltiplicare una riga per $\lambda \neq 0$. $R_i \Rightarrow \lambda \cdot R_i$.
                        \item Sostituire la riga $R_j$ con una somma. $R_j \Rightarrow R_j + \lambda \cdot R_i$.
                        \item Scambiare due righe.
                    \end{enumerate}
            \section{Matrice a scalini}
                \paragraph{}Una matrice $n \cdot m$  è detta a \textbf{a scalini} se:
                    \begin{enumerate}
                        \item Le righe sono \textbf{in fondo}.
                        \item Il primo elemento di ogni riga, se esiste, è \textbf{a destra} del primo elemento $\neq 0$ della riga precedente. Un tale elemento è detti \textbf{Pivot}.
                    \end{enumerate}
                        \begin{center}
                        $
                        \begin{bmatrix}
                            1 & 1 & 1\\
                            1 & 0 & 0\\
                            0 & 0 & 1
                        \end{bmatrix}
                        NO
                        $
                        $
                        \begin{bmatrix}
                            1 & 1 & 1\\
                            0 & 1 & 1\\
                            0 & 0 & 1
                        \end{bmatrix}
                        $
                        SI
                        $
                        \begin{bmatrix}
                            0 & 1 & 1\\
                            1 & 1 & 0\\
                            0 & 0 & 1
                        \end{bmatrix}
                        $
                        NO
                    \end{center}
            \section{Algoritmo di Gauss}
                \begin{enumerate}
                    \item Se la matrice è gia in forma a \textbf{scalini} si termina. \textbf{END}.
                    \item Si cerca il primo elemento $\neq 0$ della prima colonna $\neq 0$.
                    \item Scambiando le righe possiamo supporre che questo elemento è il \textbf{pivot} della prima riga. Lo segniamo con \textit{p}. 
                    \item Se siamo in forma a scalini si \textbf{termina}. \textbf{END}.
                    \item Si annullano tutti gli elementi della colonna di \textit{p} con operazioni di tipo $R_j \Rightarrow R_j + \lambda \cdot R_i$.
                    \item Se siamo in forma a scalini si \textbf{termina}. \textbf{END}.
                    \item Si ricomincia con la matrice ottenuta \textbf{escludendo} la prima riga.
                \end{enumerate}
                \subparagraph{Esempio}
                $$
                \begin{bmatrix}
                    \textcolor{orange}{1} & -1 & 0 & 3\\
                    3 & -1 & 1 & 10\\
                    1 & 5 & 2 & 1
                \end{bmatrix}
                $$
                \paragraph{}Il \textbf{pivot} della prima riga è 1, ora devo annullare tutti gli elementi della colonna del pivot.
                $$
                    \xrightarrow[]{R_2-3R_1}                    
                    \begin{bmatrix}
                        \textcolor{orange}{1} & -1 & 0 & 3\\
                        \textcolor{blue}{0} & 2 & 1 & 1\\
                        1 & 5 & 2 & 1
                    \end{bmatrix}
                    \xrightarrow[]{R_3-R_1}                    
                    \begin{bmatrix}
                        \textcolor{orange}{1} & -1 & 0 & 3\\
                        \textcolor{blue}{0} & 2 & 1 & 1\\
                        \textcolor{blue}{0} & 6 & 2 & -2
                    \end{bmatrix}
                $$
                La prima riga è \textbf{completata}, si ripete l'algoritmo escludendola.
                $$
                    \begin{bmatrix}
                        \textcolor{orange}{1} & -1 & 0 & 3\\
                        \hline
                        \textcolor{blue}{0} & \textcolor{orange}{2} & 1 & 1\\
                        \textcolor{blue}{0} & 6 & 2 & -2
                    \end{bmatrix}
                $$
                Nella seconda riga il \textbf{pivot} è 2, si procede annullando le colonne sotto il pivot.
                $$
                    \xrightarrow[]{R_3-R_1}
                    \begin{bmatrix}
                        \textcolor{orange}{1} & -1 & 0 & 3\\
                        \hline
                        \textcolor{blue}{0} & \textcolor{orange}{2} & 1 & 1\\
                        \textcolor{blue}{0} & 6 & 2 & -2
                    \end{bmatrix}
                    \xrightarrow[]{R_3-3R_2}
                    \begin{bmatrix}
                        \textcolor{orange}{1} & -1 & 0 & 3\\
                        \hline
                        \textcolor{blue}{0} & \textcolor{orange}{2} & 1 & 1\\
                        \textcolor{blue}{0} & \textcolor{blue}{0} & -1 & 5
                    \end{bmatrix}
                $$
                La seconda riga è \textbf{completata}, si ripete l'algoritmo escludendola.
                $$
                \begin{bmatrix}
                    \textcolor{orange}{1} & -1 & 0 & 3\\
                    \textcolor{blue}{0} & \textcolor{orange}{2} & 1 & 1\\
                    \hline
                    \textcolor{blue}{0} & \textcolor{blue}{0} & \textcolor{orange}{-1} & 5
                \end{bmatrix}
                $$
                L'algoritmo termina poiché -1 è un \textbf{pivot} e non ci sono colonne da annullare.
                \subparagraph{Conclusioni}La matrice ritrasformata in sistema di equazioni è la seguente:
                $$
                \begin{cases}
                    x_1-x_2+3x_4=0\\
                    2x_2+x_3+x_4=0\\
                    -x_3-5x_4=0
                \end{cases}
                $$
                La \textbf{colonna di $x_4$} è senza \textbf{pivot} quindi \textbf{$x_4$} è detta \textbf{variabile libera}, e può assumere qualsiasi
                valore nel sistema.
                Sostituiamo la \textbf{variabile libera} $x_4$ con il parametro \textit{t}.
                $$
                \begin{cases}
                    x_1-x_2+3t=0\\
                    2x_2+x_3+t=0\\
                    -x_3-5t=0
                \end{cases}
                \begin{cases}
                    x_1-x_2+3t=0\\
                    2x_2+x_3+t=0\\
                    x_3=-5t
                \end{cases}
                \begin{cases}
                    x_1-x_2+3t=0\\
                    2x_2-5t+t=0\\
                    x_3=-5t
                \end{cases}
                $$
                $$
                \begin{cases}
                    x_1-x_2+3t=0\\
                    x_2=2t\\
                    x_3=-5t
                \end{cases}
                \begin{cases}
                    x_1-2t+3t=0\\
                    x_2=2t\\
                    x_3=-5t
                \end{cases}
                \begin{cases}
                    x_1=-t\\
                    x_2=2t\\
                    x_3=-5t
                \end{cases}
                $$ 
            L'equazione ha \textbf{infinite soluzioni} che possono essere parametrizzate in \textit{t}.
            \subsection{Casi possibili}
            \paragraph{}Se nella forma a scalini:
                \begin{enumerate}
                    \item \textbf{Ogni colonna} "non aggiunta" \textbf{ha un pivot} $\Leftrightarrow \exists$ \textbf{unica soluzione}
                    \item C'è un \textbf{pivot nell'ultima colonna} $\Leftrightarrow \nexists$ \textbf{soluzione}
                    \item C'è una \textbf{colonna "non aggiunta" senza pivot} e l'ultima colonna non ne ha $\Leftrightarrow \exists \; \infty$ \textbf{soluzioni} 
                \end{enumerate}
        \clearpage
        \section{Matrice ridotta a scalini}
            \paragraph{}Una matrice è in forma \textbf{ridotta} a scalini se:
                \begin{itemize}
                    \item È in forma \textbf{a scalini}
                    \item Ogni \textbf{pivot} è = 1
                    \item Ogni \textbf{pivot} è l'unico elemento $\neq 0$ nella sua colonna
                \end{itemize}
                \subparagraph{Esempi}
                $$
                \begin{bmatrix}
                    \textcolor{blue}{1} & 2 & 0 & 0\\
                    0 & 0 & \textcolor{blue}1 & 0\\
                    0 & 0 & 0 & \textcolor{blue}{1}
                \end{bmatrix}
                \text{SI }
                \begin{bmatrix}
                    \textcolor{blue}{1} & 2 & \textcolor{orange}{3} & \textcolor{orange}{4}\\
                    0 & 0 & \textcolor{blue}{1} & \textcolor{orange}{2}\\
                    0 & 0 & 0 & \textcolor{blue}{1}
                \end{bmatrix}
                \text{NO (A scalini ma non ridotta)}
                $$
        \section{Algoritmo di Gauss-Jordan}
            \paragraph{}L'algoritmo produce una matrice in forma \textbf{ridotta} a scalini attraverso operazioni
            del tipo A, B, C.
                \begin{enumerate}
                    \item Con l'\textbf{algoritmo di Gauss} si riduce a scalini la matrice.
                    \item Nelle colonne dei pivot gli elementi della colonna superiore e a sinistra nella riga sono già = 0.
                    \textbf{Annullare} gli elementi sopra il pivot nella colonna con \textbf{operazioni del tipo B} ($R_j \Rightarrow R_j + \lambda \cdot R_i$).
                    \item In ogni riga si \textbf{cerca il pivot} (se esiste). Se il pivot $\lambda \neq 1$, si moltiplica la riga per $\frac{1}{\lambda}$.
                \end{enumerate}
                \subparagraph{Esempio}Partiamo da una matrice già ridotta a scalini dall'algoritmo di Gauss.
                    $$
                    \begin{bmatrix}
                        2 & 1 & -1 & \vline & -1\\
                        3 & 2 & -1 & \vline & 0\\
                        4 & -3 & 1 & \vline & -1\\
                        5 & -2 & 2 & \vline & 2
                    \end{bmatrix}
                    \xrightarrow[]{\text{Algoritmo di Gauss}}
                    \begin{bmatrix}
                        \textcolor{blue}{2} & 1 & -1 & \vline & -1\\
                        0 & \textcolor{blue}{1} & 1 & \vline & 3\\
                        0 & 0 & \textcolor{blue}{1} & \vline & 2\\
                        0 & 0 & 0 & \vline & 0
                    \end{bmatrix}
                    $$ 
                    Ora applichiamo l'\textbf{algoritmo di Gauss-Jordan} alla matrice a scalini per trasformarla in matrice \textbf{ridotta} a scalini.
                    $$
                    \begin{bmatrix}
                        \textcolor{blue}{2} & 1 & -1 & \vline & -1\\
                        0 & \textcolor{blue}{1} & 1 & \vline & 3\\
                        0 & 0 & \textcolor{blue}{1} & \vline & 2\\
                        0 & 0 & 0 & \vline & 0
                    \end{bmatrix}
                    $$
                    Si \textbf{azzerano} gli elementi nelle colonne dei pivot che sono $\neq$ 0.
                    $$
                    \begin{bmatrix}
                        2 & \textcolor{orange}{1} & \textcolor{orange}{-1} & \vline & -1\\
                        0 & 1 & \textcolor{orange}{1} & \vline & 3\\
                        0 & 0 & 1 & \vline & 2\\
                        0 & 0 & 0 & \vline & 0
                    \end{bmatrix}
                    \xrightarrow[]{R_1-R_2}
                    \begin{bmatrix}
                        2 & \textcolor{blue}{0} & \textcolor{orange}{-2} & \vline & -4\\
                        0 & 1 & \textcolor{orange}{1} & \vline & 3\\
                        0 & 0 & 1 & \vline & 2\\
                        0 & 0 & 0 & \vline & 0
                    \end{bmatrix}
                    \xrightarrow[R_2-R_3]{R_1+2R_3}
                    \begin{bmatrix}
                        2 & \textcolor{blue}{0} & \textcolor{blue}{0} & \vline & 0\\
                        0 & 1 & \textcolor{blue}{0} & \vline & 1\\
                        0 & 0 & 1 & \vline & 2\\
                        0 & 0 & 0 & \vline & 0
                    \end{bmatrix}
                    $$
                    Ora nelle colonne dei pivot \textbf{tutti gli elementi sono = 0} eccetto il pivot.
                    Si individuano i \textbf{pivot} $\neq 1$ e si procede con la loro \textbf{trasformazione a 1}.
                    Si moltiplicano le righe con i \textbf{pivot} $\neq$ 1 per il loro \textbf{reciproco}.
                    $$
                    \begin{bmatrix}
                        \textcolor{orange}{2} & 0 & 0 & \vline & 0\\
                        0 & 1 & 0 & \vline & 1\\
                        0 & 0 & 1 & \vline & 2\\
                        0 & 0 & 0 & \vline & 0
                    \end{bmatrix}
                    \xrightarrow[]{R_1\rightarrow \frac{1}{2}R_1}
                    \begin{bmatrix}
                        \textcolor{blue}{1} & 0 & 0 & \vline & 0\\
                        0 & 1 & 0 & \vline & 1\\
                        0 & 0 & 1 & \vline & 2\\
                        0 & 0 & 0 & \vline & 0
                    \end{bmatrix}
                    $$
                \subparagraph{Conclusioni}
                $$
                \begin{cases}
                x_1=0\\
                x_2=1\\
                x_3=2  
                \end{cases}
                $$
                %35:00
\end{document}