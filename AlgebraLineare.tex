\documentclass[a4paper, 12pt]{report}
\usepackage[italian]{babel}
\usepackage{graphicx}
\usepackage{amsmath}

\begin{document}
\title{
\textbf{Algebra Lineare}}
\author{Stefano Piccoli}
\date{\today}
\maketitle
\tableofcontents
	\chapter*{Introduzione}
	\addcontentsline{toc}{chapter}{Introduzione}
        \paragraph{}L'\textbf{Algebra Lineare} si occupa di trovare soluzioni ad equazioni e sistemi \textbf{lineari}.
            \begin{center} 
                $$\begin{cases}
                    E1: x+y=3\\
                    E2: x+2y=5
                \end{cases}$$            
                $E2-E1: \textbf{y=5-3=2}$\\
                Sostituzione: $\textbf{x=1}$
            
                $$\begin{cases}
                    E1: x+y=3\\
                    E2: 2x+2y=6
                \end{cases}$$
                $E2-E1:0=0$\\
                \textbf{Hanno le stesse soluzioni (infinità)}\\
            
                $$\begin{cases}
                    E1: x+y=3\\
                    E2: 2x+2y=5
                \end{cases}$$
                $E2-E1: 0=-1$\\
                \textbf{Nessuna soluzione comune} 
            \end{center}
        \paragraph{}Quindi abbiamo 1, $\infty$ o 0 soluzioni comuni. Così sarà in generale. 
        \section{Equazioni a 3 variabili}
            \paragraph{}Le soluzioni comuni di 3 equazioni lineari a 3 variabili corrispondono all'intersezione di 3 piani nello spazio tridimensionale.
            L'intersezione può essere di 3 tipi:
            \begin{itemize}
                \item Un punto (\textbf{unica soluzione})
                \item Una retta o un piano
                \item 0 (\textbf{$\infty$ soluzioni})
            \end{itemize}
        \section{Caso generale}
        \paragraph{}Un sistema di n equazioni lineari a m variabili.
        $$\begin{cases}
            a_{11}x_1+a_{12}x_2+\dots+a_{1m}x_m=b_1\\
            a_{12}x_1+a_{22}x_2+\dots+a_{2m}x_m=b_2\\
            \dots\\
            a_{n1}x_1+a_{n2}x_2+\dots+a_{nm}x_m=b_m
        \end{cases}$$
        \begin{center}
            $a_{ij},b_i\in\Re$\\
            $n,m > 0$\\
        \end{center}
        \subsection{Sistema omogeneo}   
        \paragraph{}Il sistema (E) è \textbf{omogeneo} se $b_1=b_2=$\dots$=b_n=0$ 
            $$\begin{cases}
                a_{11}x_1+a_{12}x_2+\dots+a_{1m}x_m=0\\
                a_{12}x_1+a_{22}x_2+\dots+a_{2m}x_m=0\\
                \dots\\
                a_{n1}x_1+a_{n2}x_2+\dots+a_{nm}x_m=0
            \end{cases}$$
        \subsection{Sistema omogeneo associato}
            \paragraph{}Un sistema omogeneo associato è un sistema dove la parte prima parte è uguale
            ad un altro e i coefficienti dopo l'uguale sono \textbf{0}.
        \subsection{Soluzione di un sistema}
            \paragraph{}\textbf{Soluzione di un sistema = soluzione di un caso particolare + soluzione dell'omogenea associata}.
            \paragraph{Esempio} $2x+3y=5$, $n=1, m=2$
                \subparagraph{Soluzione particolare}
                    \begin{align*}
                        &2x+3y=5\\
                        &x=y=1
                    \end{align*}
                \subparagraph{Soluzione omogenea}
                    \begin{align*}
                        &2x+3y=0\\
                        &x=-\frac{3}{2}y
                    \end{align*}
                \subparagraph{Soluzione generale}Definiamo s parametro nel ruolo di y.
                    \begin{align*}
                        &x=1+(-\frac{3}{2})s\\
                        &y=1+s
                    \end{align*}
        \subsection{Trovare soluzioni comuni}
            \paragraph{}Per trovare soluzioni comuni di E è necessario semplificare.
            Le 3 operazioni utili per semplificare sono:
                    \begin{itemize}
                        \item Moltiplicare un'equazione $E_i$ per una costante. $\lambda \neq 0$. $E_i\Rightarrow\lambda E_i$ 
                        \item Moltiplicare un'equazione $E_i$ per $\lambda \neq 0$ e fare la somma con $E_j$.\\ $Ej\Rightarrow E_j+\lambda E_i$. 
                        \item Scambiare due equazioni.
                    \end{itemize}
    
\end{document}